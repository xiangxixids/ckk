
%% bare_jrnl.tex
%% V1.4a
%% 2014/09/17
%% by Michael Shell
%% see http://www.michaelshell.org/
%% for current contact information.
%%
%% This is a skeleton file demonstrating the use of IEEEtran.cls
%% (requires IEEEtran.cls version 1.8a or later) with an IEEE
%% journal paper.
%%
%% Support sites:
%% http://www.michaelshell.org/tex/ieeetran/
%% http://www.ctan.org/tex-archive/macros/latex/contrib/IEEEtran/
%% and
%% http://www.ieee.org/

%%*************************************************************************
%% Legal Notice:
%% This code is offered as-is without any warranty either expressed or
%% implied; without even the implied warranty of MERCHANTABILITY or
%% FITNESS FOR A PARTICULAR PURPOSE!
%% User assumes all risk.
%% In no event shall IEEE or any contributor to this code be liable for
%% any damages or losses, including, but not limited to, incidental,
%% consequential, or any other damages, resulting from the use or misuse
%% of any information contained here.
%%
%% All comments are the opinions of their respective authors and are not
%% necessarily endorsed by the IEEE.
%%
%% This work is distributed under the LaTeX Project Public License (LPPL)
%% ( http://www.latex-project.org/ ) version 1.3, and may be freely used,
%% distributed and modified. A copy of the LPPL, version 1.3, is included
%% in the base LaTeX documentation of all distributions of LaTeX released
%% 2003/12/01 or later.
%% Retain all contribution notices and credits.
%% ** Modified files should be clearly indicated as such, including  **
%% ** renaming them and changing author support contact information. **
%%
%% File list of work: IEEEtran.cls, IEEEtran_HOWTO.pdf, bare_adv.tex,
%%                    bare_conf.tex, bare_jrnl.tex, bare_conf_compsoc.tex,
%%                    bare_jrnl_compsoc.tex, bare_jrnl_transmag.tex
%%*************************************************************************


% *** Authors should verify (and, if needed, correct) their LaTeX system  ***
% *** with the testflow diagnostic prior to trusting their LaTeX platform ***
% *** with production work. IEEE's font choices and paper sizes can       ***
% *** trigger bugs that do not appear when using other class files.       ***                          ***
% The testflow support page is at:
% http://www.michaelshell.org/tex/testflow/



\documentclass[journal]{IEEEtran}
%
% If IEEEtran.cls has not been installed into the LaTeX system files,
% manually specify the path to it like:
% \documentclass[journal]{../sty/IEEEtran}





% Some very useful LaTeX packages include:
% (uncomment the ones you want to load)
\usepackage{times}
\usepackage{helvet}
\usepackage{courier}
%defined by me
\usepackage{amsmath}
\usepackage{graphicx}
\usepackage{epstopdf}
\usepackage{booktabs}
\usepackage{makecell}
\usepackage{multirow}
\usepackage{graphicx,subfigure}
\usepackage{diagbox}
\usepackage{rotating}

\graphicspath{{fig/}}

% *** MISC UTILITY PACKAGES ***
%
%\usepackage{ifpdf}
% Heiko Oberdiek's ifpdf.sty is very useful if you need conditional
% compilation based on whether the output is pdf or dvi.
% usage:
% \ifpdf
%   % pdf code
% \else
%   % dvi code
% \fi
% The latest version of ifpdf.sty can be obtained from:
% http://www.ctan.org/tex-archive/macros/latex/contrib/oberdiek/
% Also, note that IEEEtran.cls V1.7 and later provides a builtin
% \ifCLASSINFOpdf conditional that works the same way.
% When switching from latex to pdflatex and vice-versa, the compiler may
% have to be run twice to clear warning/error messages.






% *** CITATION PACKAGES ***
%
%\usepackage{cite}
% cite.sty was written by Donald Arseneau
% V1.6 and later of IEEEtran pre-defines the format of the cite.sty package
% \cite{} output to follow that of IEEE. Loading the cite package will
% result in citation numbers being automatically sorted and properly
% "compressed/ranged". e.g., [1], [9], [2], [7], [5], [6] without using
% cite.sty will become [1], [2], [5]--[7], [9] using cite.sty. cite.sty's
% \cite will automatically add leading space, if needed. Use cite.sty's
% noadjust option (cite.sty V3.8 and later) if you want to turn this off
% such as if a citation ever needs to be enclosed in parenthesis.
% cite.sty is already installed on most LaTeX systems. Be sure and use
% version 5.0 (2009-03-20) and later if using hyperref.sty.
% The latest version can be obtained at:
% http://www.ctan.org/tex-archive/macros/latex/contrib/cite/
% The documentation is contained in the cite.sty file itself.






% *** GRAPHICS RELATED PACKAGES ***
%
\ifCLASSINFOpdf
  % \usepackage[pdftex]{graphicx}
  % declare the path(s) where your graphic files are
  % \graphicspath{{../pdf/}{../jpeg/}}
  % and their extensions so you won't have to specify these with
  % every instance of \includegraphics
  % \DeclareGraphicsExtensions{.pdf,.jpeg,.png}
\else
  % or other class option (dvipsone, dvipdf, if not using dvips). graphicx
  % will default to the driver specified in the system graphics.cfg if no
  % driver is specified.
  % \usepackage[dvips]{graphicx}
  % declare the path(s) where your graphic files are
  % \graphicspath{{../eps/}}
  % and their extensions so you won't have to specify these with
  % every instance of \includegraphics
  % \DeclareGraphicsExtensions{.eps}
\fi
% graphicx was written by David Carlisle and Sebastian Rahtz. It is
% required if you want graphics, photos, etc. graphicx.sty is already
% installed on most LaTeX systems. The latest version and documentation
% can be obtained at:
% http://www.ctan.org/tex-archive/macros/latex/required/graphics/
% Another good source of documentation is "Using Imported Graphics in
% LaTeX2e" by Keith Reckdahl which can be found at:
% http://www.ctan.org/tex-archive/info/epslatex/
%
% latex, and pdflatex in dvi mode, support graphics in encapsulated
% postscript (.eps) format. pdflatex in pdf mode supports graphics
% in .pdf, .jpeg, .png and .mps (metapost) formats. Users should ensure
% that all non-photo figures use a vector format (.eps, .pdf, .mps) and
% not a bitmapped formats (.jpeg, .png). IEEE frowns on bitmapped formats
% which can result in "jaggedy"/blurry rendering of lines and letters as
% well as large increases in file sizes.
%
% You can find documentation about the pdfTeX application at:
% http://www.tug.org/applications/pdftex





% *** MATH PACKAGES ***
%
%\usepackage[cmex10]{amsmath}
% A popular package from the American Mathematical Society that provides
% many useful and powerful commands for dealing with mathematics. If using
% it, be sure to load this package with the cmex10 option to ensure that
% only type 1 fonts will utilized at all point sizes. Without this option,
% it is possible that some math symbols, particularly those within
% footnotes, will be rendered in bitmap form which will result in a
% document that can not be IEEE Xplore compliant!
%
% Also, note that the amsmath package sets \interdisplaylinepenalty to 10000
% thus preventing page breaks from occurring within multiline equations. Use:
%\interdisplaylinepenalty=2500
% after loading amsmath to restore such page breaks as IEEEtran.cls normally
% does. amsmath.sty is already installed on most LaTeX systems. The latest
% version and documentation can be obtained at:
% http://www.ctan.org/tex-archive/macros/latex/required/amslatex/math/





% *** SPECIALIZED LIST PACKAGES ***
%
%\usepackage{algorithmic}
% algorithmic.sty was written by Peter Williams and Rogerio Brito.
% This package provides an algorithmic environment fo describing algorithms.
% You can use the algorithmic environment in-text or within a figure
% environment to provide for a floating algorithm. Do NOT use the algorithm
% floating environment provided by algorithm.sty (by the same authors) or
% algorithm2e.sty (by Christophe Fiorio) as IEEE does not use dedicated
% algorithm float types and packages that provide these will not provide
% correct IEEE style captions. The latest version and documentation of
% algorithmic.sty can be obtained at:
% http://www.ctan.org/tex-archive/macros/latex/contrib/algorithms/
% There is also a support site at:
% http://algorithms.berlios.de/index.html
% Also of interest may be the (relatively newer and more customizable)
% algorithmicx.sty package by Szasz Janos:
% http://www.ctan.org/tex-archive/macros/latex/contrib/algorithmicx/




% *** ALIGNMENT PACKAGES ***
%
%\usepackage{array}
% Frank Mittelbach's and David Carlisle's array.sty patches and improves
% the standard LaTeX2e array and tabular environments to provide better
% appearance and additional user controls. As the default LaTeX2e table
% generation code is lacking to the point of almost being broken with
% respect to the quality of the end results, all users are strongly
% advised to use an enhanced (at the very least that provided by array.sty)
% set of table tools. array.sty is already installed on most systems. The
% latest version and documentation can be obtained at:
% http://www.ctan.org/tex-archive/macros/latex/required/tools/


% IEEEtran contains the IEEEeqnarray family of commands that can be used to
% generate multiline equations as well as matrices, tables, etc., of high
% quality.




% *** SUBFIGURE PACKAGES ***
%\ifCLASSOPTIONcompsoc
%  \usepackage[caption=false,font=normalsize,labelfont=sf,textfont=sf]{subfig}
%\else
%  \usepackage[caption=false,font=footnotesize]{subfig}
%\fi
% subfig.sty, written by Steven Douglas Cochran, is the modern replacement
% for subfigure.sty, the latter of which is no longer maintained and is
% incompatible with some LaTeX packages including fixltx2e. However,
% subfig.sty requires and automatically loads Axel Sommerfeldt's caption.sty
% which will override IEEEtran.cls' handling of captions and this will result
% in non-IEEE style figure/table captions. To prevent this problem, be sure
% and invoke subfig.sty's "caption=false" package option (available since
% subfig.sty version 1.3, 2005/06/28) as this is will preserve IEEEtran.cls
% handling of captions.
% Note that the Computer Society format requires a larger sans serif font
% than the serif footnote size font used in traditional IEEE formatting
% and thus the need to invoke different subfig.sty package options depending
% on whether compsoc mode has been enabled.
%
% The latest version and documentation of subfig.sty can be obtained at:
% http://www.ctan.org/tex-archive/macros/latex/contrib/subfig/




% *** FLOAT PACKAGES ***
%
%\usepackage{fixltx2e}
% fixltx2e, the successor to the earlier fix2col.sty, was written by
% Frank Mittelbach and David Carlisle. This package corrects a few problems
% in the LaTeX2e kernel, the most notable of which is that in current
% LaTeX2e releases, the ordering of single and double column floats is not
% guaranteed to be preserved. Thus, an unpatched LaTeX2e can allow a
% single column figure to be placed prior to an earlier double column
% figure. The latest version and documentation can be found at:
% http://www.ctan.org/tex-archive/macros/latex/base/


%\usepackage{stfloats}
% stfloats.sty was written by Sigitas Tolusis. This package gives LaTeX2e
% the ability to do double column floats at the bottom of the page as well
% as the top. (e.g., "\begin{figure*}[!b]" is not normally possible in
% LaTeX2e). It also provides a command:
%\fnbelowfloat
% to enable the placement of footnotes below bottom floats (the standard
% LaTeX2e kernel puts them above bottom floats). This is an invasive package
% which rewrites many portions of the LaTeX2e float routines. It may not work
% with other packages that modify the LaTeX2e float routines. The latest
% version and documentation can be obtained at:
% http://www.ctan.org/tex-archive/macros/latex/contrib/sttools/
% Do not use the stfloats baselinefloat ability as IEEE does not allow
% \baselineskip to stretch. Authors submitting work to the IEEE should note
% that IEEE rarely uses double column equations and that authors should try
% to avoid such use. Do not be tempted to use the cuted.sty or midfloat.sty
% packages (also by Sigitas Tolusis) as IEEE does not format its papers in
% such ways.
% Do not attempt to use stfloats with fixltx2e as they are incompatible.
% Instead, use Morten Hogholm'a dblfloatfix which combines the features
% of both fixltx2e and stfloats:
%
% \usepackage{dblfloatfix}
% The latest version can be found at:
% http://www.ctan.org/tex-archive/macros/latex/contrib/dblfloatfix/




%\ifCLASSOPTIONcaptionsoff
%  \usepackage[nomarkers]{endfloat}
% \let\MYoriglatexcaption\caption
% \renewcommand{\caption}[2][\relax]{\MYoriglatexcaption[#2]{#2}}
%\fi
% endfloat.sty was written by James Darrell McCauley, Jeff Goldberg and
% Axel Sommerfeldt. This package may be useful when used in conjunction with
% IEEEtran.cls'  captionsoff option. Some IEEE journals/societies require that
% submissions have lists of figures/tables at the end of the paper and that
% figures/tables without any captions are placed on a page by themselves at
% the end of the document. If needed, the draftcls IEEEtran class option or
% \CLASSINPUTbaselinestretch interface can be used to increase the line
% spacing as well. Be sure and use the nomarkers option of endfloat to
% prevent endfloat from "marking" where the figures would have been placed
% in the text. The two hack lines of code above are a slight modification of
% that suggested by in the endfloat docs (section 8.4.1) to ensure that
% the full captions always appear in the list of figures/tables - even if
% the user used the short optional argument of \caption[]{}.
% IEEE papers do not typically make use of \caption[]'s optional argument,
% so this should not be an issue. A similar trick can be used to disable
% captions of packages such as subfig.sty that lack options to turn off
% the subcaptions:
% For subfig.sty:
% \let\MYorigsubfloat\subfloat
% \renewcommand{\subfloat}[2][\relax]{\MYorigsubfloat[]{#2}}
% However, the above trick will not work if both optional arguments of
% the \subfloat command are used. Furthermore, there needs to be a
% description of each subfigure *somewhere* and endfloat does not add
% subfigure captions to its list of figures. Thus, the best approach is to
% avoid the use of subfigure captions (many IEEE journals avoid them anyway)
% and instead reference/explain all the subfigures within the main caption.
% The latest version of endfloat.sty and its documentation can obtained at:
% http://www.ctan.org/tex-archive/macros/latex/contrib/endfloat/
%
% The IEEEtran \ifCLASSOPTIONcaptionsoff conditional can also be used
% later in the document, say, to conditionally put the References on a
% page by themselves.




% *** PDF, URL AND HYPERLINK PACKAGES ***
%
%\usepackage{url}
% url.sty was written by Donald Arseneau. It provides better support for
% handling and breaking URLs. url.sty is already installed on most LaTeX
% systems. The latest version and documentation can be obtained at:
% http://www.ctan.org/tex-archive/macros/latex/contrib/url/
% Basically, \url{my_url_here}.




% *** Do not adjust lengths that control margins, column widths, etc. ***
% *** Do not use packages that alter fonts (such as pslatex).         ***
% There should be no need to do such things with IEEEtran.cls V1.6 and later.
% (Unless specifically asked to do so by the journal or conference you plan
% to submit to, of course. )


% correct bad hyphenation here
\hyphenation{op-tical net-works semi-conduc-tor}


\begin{document}
%
% paper title
% Titles are generally capitalized except for words such as a, an, and, as,
% at, but, by, for, in, nor, of, on, or, the, to and up, which are usually
% not capitalized unless they are the first or last word of the title.
% Linebreaks \\ can be used within to get better formatting as desired.
% Do not put math or special symbols in the title.
\title{Multi-scale Quantum Harmonic Oscillator Optimization Algorithm}
%
%
% author names and IEEE memberships
% note positions of commas and nonbreaking spaces ( ~ ) LaTeX will not break
% a structure at a ~ so this keeps an author's name from being broken across
% two lines.
% use \thanks{} to gain access to the first footnote area
% a separate \thanks must be used for each paragraph as LaTeX2e's \thanks
% was not built to handle multiple paragraphs
%

\author{Peng~Wang%,~\IEEEmembership{Member,~IEEE,}
        %John~Doe,~\IEEEmembership{Fellow,~OSA,}
        %and~Jane~Doe,~\IEEEmembership{Life~Fellow,~IEEE}% <-this % stops a space
\thanks{Peng Wang with the Department of Electrical and Computer Engineering, Georgia Institute of Technology, Chengdu, GA, 610225 China (e-mail: wp002005@163.com).}}%
%\thanks{Manuscript received April 19, 2005; revised September 17, 2014.}}

% note the % following the last \IEEEmembership and also \thanks -
% these prevent an unwanted space from occurring between the last author name
% and the end of the author line. i.e., if you had this:
%
% \author{....lastname \thanks{...} \thanks{...} }
%                     ^------------^------------^----Do not want these spaces!
%
% a space would be appended to the last name and could cause every name on that
% line to be shifted left slightly. This is one of those "LaTeX things". For
% instance, "\textbf{A} \textbf{B}" will typeset as "A B" not "AB". To get
% "AB" then you have to do: "\textbf{A}\textbf{B}"
% \thanks is no different in this regard, so shield the last } of each \thanks
% that ends a line with a % and do not let a space in before the next \thanks.
% Spaces after \IEEEmembership other than the last one are OK (and needed) as
% you are supposed to have spaces between the names. For what it is worth,
% this is a minor point as most people would not even notice if the said evil
% space somehow managed to creep in.



% The paper headers
\markboth{Journal of \LaTeX\ Class Files,~Vol.~13, No.~9, September~2014}%
{Shell \MakeLowercase{\textit{et al.}}: Bare Demo of IEEEtran.cls for Journals}
% The only time the second header will appear is for the odd numbered pages
% after the title page when using the twoside option.
%
% *** Note that you probably will NOT want to include the author's ***
% *** name in the headers of peer review papers.                   ***
% You can use \ifCLASSOPTIONpeerreview for conditional compilation here if
% you desire.




% If you want to put a publisher's ID mark on the page you can do it like
% this:
%\IEEEpubid{0000--0000/00\$00.00~\copyright~2014 IEEE}
% Remember, if you use this you must call \IEEEpubidadjcol in the second
% column for its text to clear the IEEEpubid mark.



% use for special paper notices
%\IEEEspecialpapernotice{(Invited Paper)}




% make the title area
\maketitle

% As a general rule, do not put math, special symbols or citations
% in the abstract or keywords.
\begin{abstract}
Multi-scale Quantum Harmonic Oscillator Optimization Algorithm (MQHOA) is proposed and detail researched in this paper. It is a new optimization algorithm base on quantum theory. The objective function is viewed as the potential energy of a hypothetical bound states quantum system by MQHOA. The harmonic oscillator potential is used to approximate the objective function by Taylor approximation. The wavefunctions of quantum harmonic oscillator are used to construct the sampling probability density functions. Some important theoretical problems, such as quantum tunnel effect, zero energy, uncertainty relation, two scale relation et al, are proved and researched in this paper. Wavefunction, tunnel effect, zero energy and uncertainty relation reflect the quantum effects of MQHOA. The uncertainty relation implies that only one scale iteration can't get the results accurately. So MQHOA include two iteration processes: Quantum Harmonic Oscillator (QHO) iteration process and Multi-scale (M) iteration process. The experiments have proved that MQHOA is an accurate optimization algorithm. Its convergence process can suit different objective functions automatically.
\end{abstract}

% Note that keywords are not normally used for peerreview papers.
\begin{IEEEkeywords}
IEEEtran, journal, \LaTeX, paper, template.
\end{IEEEkeywords}






% For peer review papers, you can put extra information on the cover
% page as needed:
% \ifCLASSOPTIONpeerreview
% \begin{center} \bfseries EDICS Category: 3-BBND \end{center}
% \fi
%
% For peerreview papers, this IEEEtran command inserts a page break and
% creates the second title. It will be ignored for other modes.
\IEEEpeerreviewmaketitle





\section{Introduction}
\IEEEPARstart{M}{ulti-scale} Quantum Harmonic Oscillator Optimization Algorithm (MQHOA) is a novel optimization algorithm which is proposed in 2013\cite{wangpeng2013}. In previous work experiments and analyses are done for 15 typical two-dimensional test functions(Benchmark functions: Schaffer, Ackley, Levy, Matyas, Griewank, Easom, Beale, Bohachevsky, Booth, Michalewics, Rastrigrin, Sphere, Rosenbrock, Sum Square, Zakharov). The experiments have shown that the basic MQHOA has a powerful performance in finding global optima.\cite{wangpeng2013}. The population parameter and sampling parameter are researched in \cite{wangpeng2013}. The uncertainty principle, zero energy and quantum tunnel effect of MQHOA are also researched by \cite{WangForthcoming}. As a kind of new optimization algorithm there is a lot of problems need to research. This paper will conduct detail research on MQHOA's physics model.

MQHOA was inspired by the wavefunction of quantum harmonic oscillator. It tranforms the optimization problems to find the low energy state of potential $V(x)=f(x)$. The complex objective function's second order Taylor approximation is Harmonic oscillator potential. So according to quantum theory the wavefunction of quantum harmonic oscillator reflects the distribution of optimal solution \cite{Griffiths2006}.

In recent years quantum model has been often used to construct new optimization algorithms \cite{Kosztin1997}. The famous one is quantum annealing algorithm (QA). In \cite{Lorenzo2005} QA is used to optimize three simple cases(harmonic function, double well function, parabolic washboard function). QA is also used to solve the traveling salesman problem successfully\cite{Martonak2004}.


    The framework of The basic MQHOA




an optimization problem is the problem of finding the objective function's best solution  $f(x_{opt})$ (minimum value) from all feasible solutions. It is important in almost every branch of science. Most of optimization algorithm also can be used to solve combinatorial optimization problems. Quantum theory is an important model which is often used to construct a new optimization algorithm. We know that the major feature of optimization is random, so it is convenient to view the optimization process as a kind of quantum process. Ordinarily the objective function has many local minima. Using quantum fluctuations the optimization algorithm can escape the local minimum positions and find the absolute minimum position. The wavefunction acts as an important role to describe the position of particles in quantum theory. Because the wavefunction is probability wave it can describe the probability distribution of particles. In quantum physics harmonic oscillator potential is used to approximate the complex oscillator. In this paper we also use harmonic oscillator to approximate the complex objective function.

The most famous optimization algorithm based quantum theory is quantum annealing algorithm(QA). QA assumes that the objective function is the potential energy function in Schr\"odinger equation($V(x)=f(x)$). So the Hamiltonian has the form:

\begin{equation}
H=-\frac{\hbar}{2m_t}\nabla^2+V(x)
\end{equation}

This is the quantum bound state system under potential energy $V(x)$. Unfortunately only simple potential $V(x)$ can be solved analytically. The annealing process increases $m_t$ from small mass to infinite mass gradually. The quantum fluctuations are reduced with particle mass increasing. This does not comply with physical meaning.  We consider that annealing process ought to decrease from high energy state to ground state. During the quantum annealing process tunnel effect is available to avoid trapping into local optimal areas. Diffusion Monte Carlo (DMC) method is used to find the ground state of bound state system. The diffusion equation and random walker are used by DMC. No knowledge of the wavefunction is required. \cite{Kosztin1997}. In \cite{Lorenzo2005} quantum annealing is used to optimize three simple cases(harmonic function, double well function, parabolic washboard function). QA is also used to solve the traveling salesman problem successfully\cite{Martonak2004}.

In recent years another algorithm based on quantum model is proposed. It is named Quantum Particle Swarm Optimization(QPSO)\cite{Sun2004}\cite{sunjun2012}. QPSO uses delta potential well to approach the objective function. The ground state wavefunctions(Laplace distribution) of the quantum system bounded by delta potential are used as sampling function to search global minimum. In essence QPSO is a kind of quantum annealing algorithm. Characteristic length acts as the role of annealing parameter. The convergence speed of QPSO is very fast. Harmonic oscillator well and square well have been also used to construct QPSO to solve economic load dispatch problem and Electromagnetics problem etc\cite{Li2012}.


Constructing a bound state quantum system with potential well $f(x)$ is a very important and interesting idea of QA and QPSO. The probability mechanism of quantum system plays an important role. Inspired by this idea Multi-scale Quantum Harmonic Oscillator Algorithm(MQHOA) has first been proposed and researched in \cite{wangpeng2013}. MQHOA also assumes that the objective function $f(x)$ is potential well $V(x)$ in Schr\"odinger equation. According to this hypothesis the optimization problems are transformed into finding the minimum of the quantum system's potential energy. Unlike QA and QPSO, MQHOA uses harmonic oscillator potential to approximate the objective functions at different scale. Because harmonic oscillator potential is the Taylor series' two-order approximation of arbitrary objective function. The wavefunction of harmonic oscillator is used as sampling probability density function by MQHOA. MQHOA's annealing process is from high energy state to ground state. This is also different from QA and QPSO. $k$ Gauss sampling areas are used to simulate the high energy state wavefunctions. Gauss sampling is the basic operator in this algorithm. It don't need to solve Schr\"odinger equation. In this paper annealing process of MQHOA is called Quantum Harmonic Oscillator process(QHO process). In order to find the global minimum different scale wavefunctions of harmonic oscillator are used by MQHOA. This is so-called Multi-scale process(M process). It corresponds to QA's annealing process.

MQHOA has elegant structure. It only includes two iteration processes: QHO process and M process. It also has stable algorithm parameters. In previous works 15 benchmark functions (include high dimension functions) have been tested by MQHOA successfully\cite{WangForthcoming}. MQHOA has been proved as a kind of high performance optimization algorithm. But its theoretical foundation has not been further studied. In this paper some important theoretical points are researched.

\section{Quantum Model}
The objective function $f(x)$ is seen as the potential energy of a hypothetical quantum physical system by MQHOA. The goal of optimization problem is searching the lowest energy position $f(x_{opt})$ (where $x_{opt}$ is global minimum position). Quantum harmonic oscillator process (QHO process) simulates the quantum harmonic oscillator annealing from high energy level to ground state.
\subsection{Constructing Quantum System}

Inspired by quantum annealing method the optimization problem is a king of bound states quantum system. The potential well of bound states quantum system is objective function $f(x)$. Bounded by the potential well $f(x)$  the searching process is described by the corresponding wavefunctions $\psi(x)$. This model considers the objective function $f(x)$ as potential well $V(x)$ of bound states quantum system. According to this model the Schr\"odinger equation can be written as

\begin{equation}
 \left(-\frac{\hbar^2}{2m}\frac{\partial^2}{\partial{x}^2}+f\left(x\right)\right)\psi(x)=E\psi(x)
\end{equation}

The quantum theory has told us the wavefunction $\psi(x)$ is probability distribution function. The goal of QHO is to find the wavefunction $\psi_0(x)$ of ground state. In order to find $\psi_0(x)$ annealing process from high energy level to ground state is needed. When the system is in ground state we can find the optimal position with high probability. Because the absolute minimum energy of system is $f(x_{opt})$. The energy levels are decided by objective function. In order to reduce the energy level of system $k$ optimal positions $x_i$ are retained in every iteration. This process will be continued until the wavefunction $\psi_0(x)$ of ground state is found.

\subsection{Taylor Series Approximation of Objective Function}


The Schr\"odinger equation is difficult to solve unless some simple potential (e.g. harmonic oscillator potential). Ordinarily the objective function $f(x)$  is very complex. Some appropriate simple potentials are used to approximate the objective functions. In quantum theory any potential is approximately parabolic(harmonic oscillator potential), in the neighborhood of a local minimum.\cite{Griffiths2006}
We expand $V(x)$ in a Taylor series about the local minimum $x_0$:
\begin{align}
  V(x)=V(x_0+(x-x_0))\notag \\
  =V(x_0)+V'(x_0)(x-x_0)\notag \\
  +\frac{1}{2}V''(x_0)(x-x_0)^2+\ldots
\end{align}
$V(x_0)$ is constant(you can add a constant to $V(x)$ with impunity, since that doesn't change the force). Assume $x_0$ is local minimum, so $V'(x_0)=0$ and $V''(x_0)>0$, drop the higher-order term(which are negligible as long as $(x-x_0)$ stays small), we get

\begin{equation}
V(x)\approx\frac{1}{2}V''(x_0)(x-x_0)^2
\end{equation}

Where $V''(x_0)$ is spring constant $K$. This implies that the harmonic oscillator potential is a kind of suitable simple potential to approximate the complex objective function at the local minimum position. In fact harmonic oscillator potential is also used to approximate the complex vibration of molecule by quantum theory.

\subsection{QHO Process}

Fortunately the wavefunction of harmonic oscillator has been solved. According to Schr\"odinger equation the wavefunction of harmonic oscillator potential well is
\begin{equation}
\psi_n(x)=A_n{\exp(-a^2x^2/2)}H_n(ax)
\end{equation}
where $A_n$ is normalization constant, $a=\sqrt{m\omega/\hbar}$, $x_0=0$. Note that the wavefunctions for higher $n$ have more ``humps'' within the potential well. This corresponds to a shorter wavelength and therefore by the deBroglie relationship they may be seen to have a higher momentum and therefore higher energy.

If the system is in high energy state the probability distribution of wavefunction is superposed by some Gauss distribution. With the energy reduced the wavefunction is convergenced to single Gauss distribution. The square of the wavefunction gives the probability of finding the oscillator at a particular value of $x$ . So the square of ground state harmonic oscillator wavefunction is

\begin{equation}
|\psi_0(x)|^2=\frac{a}{\sqrt{\pi}}{\exp{(-a^2x^2)}}
\end{equation}

It can also be written as $N(0,\sigma^2_s)$. Where  $\sigma^2_s=\frac{\hbar}{m\omega}$.
For every iteration we retain $k$ optimal positions $x_i$  as the centre of Gauss sampling to reduce the system��s energy level. When the energy level reaches the ground state we can find the position of local minimum with the probability distribution $N(\overline x_i,\sigma^2_s)$  in objective function's definition domain. $\overline x_i$ is the mean value of $k$ optimal positions $x_i$.

So the QHO process is described as follow:

Using harmonic oscillator potential well $\frac{1}{2}Kx$  to approximate the objective function $f(x)$, the corresponding wavefunction  $\psi_n(x)$ is the probability distribution of optimal position. So  $\psi_n(x)$ is the sampling position probability distribution. QHO process uses $k$  Gauss sampling areas ($N(x_i,\sigma^2_s)$ ) to construct a high energy state wavefunction. The wavefunction at scale $\sigma_s$ is defined by
\begin{equation}
\psi^s(x)=\frac{1}{k}\sum\limits_{i=1}^k\frac{1}{\sqrt{2\pi}\sigma_s}e^{-\frac{(x-x_i)^2}{2\sigma^2_s}}
\end{equation}
QHO process keep $k$ optimal positions as sampling center to reduce the system's energy at each iteration. When the standard deviation of $k$ Gauss sampling central position  $x_i$ are convergenced ($\sigma_k\leq\sigma_s$), the ground state wavefunction at scale $\sigma_s$ is found:

\begin{equation}
\psi^s_0(x)\approx\frac{1}{\sqrt{2\pi}\sigma_s}e^{-\frac{(x-\overline{x}_i)^2}{2\sigma^2_s}}
\end{equation}

Then we say the QHO process has convergenced at scale $\sigma_s$ .

\subsection{Zero Energy}
In quantum system the energy of ground state is not equal to zero. Zero energy is an important quantum effect. The physics model of MQHOA considers the objective function $f(x)$ as the potential energy $V(x)$ of Schr\"odinger equation. At scale $\sigma_s$ MQHOA also has zero energy($E_0^{\sigma_s}$). The zero energy of MQHOA at scale $\sigma_s$ can be defined as follow:

\begin{equation}
E^{\sigma_s}_0=\int{\psi^s_0(x)f(x)dx}-f(x_{opt})
\end{equation}

The zero energy exists in quantum system widely. It is a kind of quantum  phenomena. Ordinarily, the zero energy of MQHOA is proportional to $\sigma_s$. The zero energy is reduced with the scale decreasing.
Ordinarily a simplified calculation method is used to get zero energy at scale $\sigma_s$:

\begin{equation}
E^{\sigma_s}_0=f(x'_{opt})-f(x_{opt})
\end{equation}

Where $x'_{opt}$ is the best solution when the QHO process has converged at scale $\sigma_s$.


\subsection{Quantum Tunnel Effect of QHO Process}
In \cite{Levy1985} Tunneling algorithm uses tunneling phase to jump out the local minimum. MQHOA uses wavefunction to implement tunnel effect automatically. Unlike classic harmonic oscillator we can find quantum harmonic oscillator at classic forbidden areas. This is so-called quantum tunnel effect. The probability character is the nature of wavefunction. So QHO process can utilize quantum tunnel effect to avoid trap local optimal positions. For arbitrary interval $[a,b]$  the probability of quantum tunnel effect can be written this way:

\begin{equation}
p=\int^b_a{\frac{1}{k}\sum\limits_{i=1}^k\frac{1}{\sqrt{2\pi}\sigma_s}e^{-\frac{(x-x_i)^2}{2\sigma^2_s}}}
\end{equation}

This equation implies that the larger the scale is the more obvious the tunnel effect is. QHO with large scale $\sigma_s$  corresponding to global searching and the small one to local searching.

\subsection{Distribution at Scale $\sigma_s$}

In QHO process Gauss distribution acts as an important role. Why the Gauss distribution is selected by MQHOA? The physical model is only one reason. The optimal solution distribution also satisfies Gauss distribution. In this section we will analysis the optimal solution distribution at scale $\sigma_s$.

Inspired by Landon's work\cite{T.2003} , we suppose the probability distribution of optimal solution $x_{opt}$  at scale $\sigma_s$ is $p(x_{opt}|\sigma_s)$, $\varepsilon$  is sampling error. So the actual sampling position is $x=x_{opt}+\varepsilon$. Where $\varepsilon$ is small compared to $\sigma_s$, and has a probability distribution $q(\varepsilon)$, independent of $p(x_{opt}|\sigma_s)$. The optimal solution $x_{opt}=x-\varepsilon$. By the product and sum rules of probability theory, the new probability distribution is the convolution
\begin{equation}
p_{\sigma_s}(x)=\int{d\varepsilon{p}(x-\varepsilon|\sigma_s)q(\varepsilon)}
\end{equation}
Expanding this by Taylor Polynomial in powers of the $\varepsilon$ ��we have
\begin{align}
p_{\sigma_s}(x)=p(x_{opt}|\sigma_s)-\frac{\partial{p}(x_{opt}|\sigma_s)}{\partial{x_{opt}}}\int\varepsilon{q}(\varepsilon)d\varepsilon \notag\\
+\frac{1}{2}\frac{\partial^2{p}(x_{opt}|\sigma_s)}{\partial{x^2_{opt}}}\int\varepsilon^2{q}(\varepsilon)d\varepsilon+\ldots
\end{align}

Now writing for brevity $p\equiv{p}(x_{opt}|\sigma_s)$

\begin{equation}
p_{\sigma_s}(x)=p-\frac{\partial{p}}{\partial{x_{opt}}}\overline{\varepsilon}+\frac{1}{2}\frac{\partial^2{p}}{\partial{x^2_{opt}}}\overline\varepsilon^2+o(\overline\varepsilon^2) \label{eq:1}
\end{equation}

The $\varepsilon$ is as likely to be positive as negative. So $\overline\varepsilon=\int\varepsilon{q}(\varepsilon)d\varepsilon=0$

The expectation of $x^2$ is $\sigma^2_s+\varepsilon^2$. and $p_{\sigma_s}(x)=p(x|\sigma^2+\overline\varepsilon^2)$. So
\begin{equation}
p_{\sigma_s}(x)=p+\overline\varepsilon^2\frac{\partial{p}}{\partial\sigma^2_s}+o(\overline\varepsilon^2)\label{eq:2}
\end{equation}

Comparing (\ref{eq:1}) and (\ref{eq:2}) we get
\begin{equation}
\frac{\partial{p}}{\partial\sigma^2_s}=\frac{1}{2}\frac{\partial^2{p}}{\partial{x^2_{opt}}}
\end{equation}

This is a differential equation(the diffusion equation), if the initial condition is $p(x_{opt}|\sigma_s)=\delta(x_{opt})$, this initial condition correspond to find the global optimal solution accurately, and the optimal position is $x_{opt}=0$

The solution is
\begin{equation}
p(x_{opt}|\sigma_s)=\frac{1}{\sqrt{2\pi}\sigma^2_s}\exp(-\frac{x^2_{opt}}{2\sigma^2_s})
\end{equation}

It indicates that for arbitrary $q(\varepsilon)$ the probability distribution of optimal solution at scale $\sigma_s$  is normal distribution. This result proves that physics model of QHO accords with probability theory. Using Gauss function as basic sampling distribution is selected by physics model and probability model.


\section{The Framework of MQHOA}

\subsection{The Description of MQHOA}

Because of the perfect physics model and mathematical structure the framework of MQHOA is elegant. It only includes two nested iteration processes: QHO process and M process. The convergence condition of QHO process and M process is $\sigma_k\leq\sigma_s$ and $\sigma_s\leq\sigma_{min}$  respectively. Where $\sigma_k=\sqrt{\sum\limits^k_{i=1}(x_i-\overline{x}_i)^2}$. The MQHOA's framework is described as follow:


\begin{tabular}{p{22em}}
\toprule
The framework of MQHOA algorithm\\
\midrule
1:Initialization{:}$k$, $n$, $\sigma_{min}$, $\sigma_s=max-min$, $\lambda=2$ \\
2:Generate  $k$ random positions $x_i$ in objective function's definition domain $[min,max]$ \\
3:\textbf{Do}\\
4:\quad \textbf{Do}\\
5:\qquad Generate $n$ normal distribution $N(x_i,\sigma^2_s)$ positions for every $x_i$ respectively.\\
6:\qquad Select $k$ optimal positions $x_i^{opt}$ from $k\times{n}$  sampling positions according to objective function\\
7:\qquad Update $x_i$:$x_i\gets x_i^{opt}$\\
8:\qquad Calculate the standard deviation $\sigma_k$\\
9:\quad \textbf{WHILE}($\sigma_k>\sigma_s$)\\
10:\quad $\sigma_s=\frac{\sigma_s}{\lambda}$\\
11:\textbf{WHILE}($\sigma_s>\sigma_{min}$)\\
12:\textbf{OUTPUT} $x'_{opt}$\\
\bottomrule
\end{tabular}


Initial value of scale $\sigma_s$ is definition domain length. $x'_{opt}$ is the optimal position the algorithm find. $x^{opt}_i$ is $k$ optimal positions selected from  $k\times{n}$ sampling positions. $k\times{n}$ sampling positions are needed by every iteration. $k$ optimal positions are stored in $x_i$. MQHOA framework only do one thing: sampling according to wavefunction at different scale $\sigma_s$ . This is so-called ``importance sampling''.\cite{Robert2004}The halt condition $\sigma_{min}$  is defined in advance. $\sigma_{min}$ decides the optimization accuracy. In the framework there are only two parameters ($k$ and $n$) need to decide. The previous experiments \cite{wangpeng2013} have proved that  $k=20$ ,$n=200$  are suitable for most benchmark functions. Ordinary we don't need to change the values of $k$ ,$n$ and $\lambda$ . The high dimensional problems are the same as 2 dimensional problems.

In framework the superposition of $k$ Gauss sampling areas constructs the wavefunction of MQHOA. In order to reduce the energy of system $k$ optimal positions $x_i^{opt}$ are retained from $k\times{n}$ sampling positions. The QHO process $\sigma_s=\frac{\sigma_s}{\lambda}$ transforms the system from ground state at scale $\sigma_s$ to high energy state at scale $\frac{\sigma_s}{\lambda}$.

For high dimensional test function MQHOA uses two dimensional array $x_{ij}$ to store the $k$ high dimensional central positions of Gauss sampling area($N(x_{ij},\sigma_s)$). Where $i$ is dimension, $j$ is the number of Gauss sampling areas. For every dimension MQHOA calculates the value of $\sigma_k$. Until every dimension satisfies $\sigma_k\leq\sigma_s$ the QHO process at scale $\sigma_s$ is finished.

Comparing to PSO, MQHOA has simple iteration construct. MQHOA is not a kind of swarm algorithm though it looks like using Gauss sampling swarm. MQHOA only remenber $k$ global optimal positions $x_i$. For every sampling area it don't need to remember the optimal positions and velocity like PSO.

\subsection{Interpretation of the Framework}

We use benchmark test function Rastrigin to interpret the framework of MQHOA. Two dimensional Rastrigin function can be written as follow:

\begin{equation}
f(x)=x^2-10.0cos(2{\pi}x)+10.0
\end{equation}

\begin{figure*}
\centering
\subfigure[objective function and potential]{\includegraphics[width=4cm,angle=-90]{quantum.eps} }
\subfigure[large scale wavefunction]{\includegraphics[width=4cm,angle=-90]{quantum1.eps} }
\subfigure[small scale wavefunction]{\includegraphics[width=4cm,angle=-90]{quantum2.eps} }
\caption{The Wavefunction at Different Scale.}
\label{fig:sampling}
\end{figure*}


In Figure \ref{fig:sampling}(a) we find that Rastrigin function has two scale structures. The large scale structure is $x^2$ and the small one is $cos(2{\pi}x)$. In order to find the global minimum position the algorithm must find the large scale minimum position and the small scale minimum position at the same time. According to Taylor Series approximation MQHOA uses different scale harmonic oscillator potentials to approximate the two scale structures of Rastrigin. For Rastrigin function only two harmonic oscillator potentials $V1(x)$ and $V2(x)$ are the approximation of objective function. $V1(x)$ is large scale Taylor approximate and $V2(x)$ is small scale Taylor approximate. For other complex objective function MQHOA need more harmonic oscillator potentials to approximate the objective function. So MQHOA uses mutli-scale process to approximate the objective function at the different scale.

Figure \ref{fig:sampling}(b) describes the quantum wavefunction of large scale potential $V1(x)$. According  to the probability interpretation of quantum wavefunction MQHOA considers that wavefunction is probability distribution of objective function's global minimum. So the wavefunction can be seen as sampling probability function. The wavefunction of harmonic oscillator includes different energy level. $E0$ corresponds to ground state wavefuncton. The ground state wavefunction is normal distribution. $E3$ corresponds to high energy level wavefunction($E0<E1<E2<E3$). It is superposed by several normal distributions. From high energy level to gound state the wavefunction convergences to normal distribution. The QHO process of MQHOA is inspired by this process. When $\sigma_k\leq\sigma_s$ the wavefunction of MQHOA is ground state wavefunction. The QHO process at scale $\sigma_s$ is halt. Ordinarily the tunnel effect of large scale wavefunction is obvious. Large scale wavefunction corresponds to global searching.

Figure \ref{fig:sampling}(c) describes the wavefunction of small scale potential $V2(x)$. When M process of MQHOA reduces the scale $\sigma_S$ the wavefunction of MQHOA are became from ground state of large scale to high energy level state of small scale. With the scale reduced the sampling areas are convergenced to small areas. Ordinarily only two scale wavefunctions are not enough to find the global minimum, so MQHOA reduces the scale according to the equation $\sigma_s=\sigma_s/2$. This is so call multi-scale process.



\subsection{Global Multimodal gm-MQHOA}

\begin{tabular}{p{22em}}
\toprule
The framework of Multimodal MQHOA algorithm\\
\midrule
1:Initialization{:}$k$, $n$, $\sigma_{min}$, $\sigma_s=max-min$, $\lambda=2$ \\
2:Generate  $k$ random positions $x_i$ in objective function's definition domain $[min,max]$ \\
3:\textbf{Do}\\
4:\quad$\sigma'_k=\sigma_s$\\
5:\quad \textbf{Do}\\
6:\qquad Generate $n$ normal distribution $N(x_i,\sigma^2_s)$ positions for every $x_i$ respectively.\\
7:\qquad Select $k$ optimal positions $x_i^{opt}$ from $k\times{n}$  sampling positions according to objective function\\
8:\qquad Update $x_i$:$x_i\gets x_i^{opt}$\\
9:\qquad Calculate the standard deviation $\sigma_k$\\
10:\qquad $\Delta\sigma=|\sigma'_k-\sigma_k|$\\
11:\qquad $\sigma'_k=\sigma_k$\\
12:\quad \textbf{WHILE}($\sigma_k>\sigma_s$ and $\Delta\sigma{>}\sigma_s$)\\
13:\quad $\sigma_s=\frac{\sigma_s}{\lambda}$\\
14:\textbf{WHILE}($\sigma_s>\sigma_{min}$)\\
15:\textbf{OUTPUT} $x^{opt}_i$\\
\bottomrule
\end{tabular}



\section{Experiment Results}



\subsection{Function optimization results using MQHOA, QPSO and SA}
MQHOA, QPSO and SA are similar. All of them use physics model to construct optimization. In this paper we select five typical complex benchmark functions as test functions. For every function 30 experiments were carried out. The results are listed in Table \ref{tab:experiment1}. ``D'' is dimensions of benchmark functions; ``S1'' is the success numbers with accuracy level $S<0.1$ (where $S=|f(x'_{opt})-f(x_{opt})|$). ``S2'' is the success numbers with accuracy level $S<0.001$. ``S3'' is the success numbers with accuracy level $S<0.000001$. $\overline{\epsilon}$ is the average minimum function value of 30 experiments.


 Table \ref{tab:experiment1} shows that MQHOA is more accurate than QPSO and SA. As the dimensions increasing the MQHOA keeps the high accuracy but the mean computation time increases at the same time. The average running time of QPSO is not sensitive to function's dimensions but the accuracy decreases quickly as the dimension increasing. The experiment results prove that MQHOA is a self adaptive algorithm. When the function's dimensions have increased the mean computation time will increase too. The accuracy of MQHOA is best than QPSO and SA. But the convergence rate of QPSO is best than MQHOA and SA.

\begin{table*}
\centering
\caption{The function optimization results using MQHOA, QPSO and SA. For every function the experiment is repeated 30 times. $\overline{\epsilon}$ and $\overline{t}$ are the average value of 30 experiment. $\overline{t}$ is average running time. MQHOA($k=20$,$n=200$,$\sigma_{min}=0.000001$)��QPSO(The domain of contraction and expansion coefficient is $[0.5,1]$. The population coefficient is 80.) SA(The max iteration times of isothermal process is 10000. Cooling rate is 0.95. Initial temperature is 100.)}
\label{tab:experiment1}
\begin{tabular}{ccccccccccccccccc}
\toprule
\multirowcell{2}{Test \\Function}&\multirow{2}*{D}&\multicolumn{5}{c}{MQHOA}&\multicolumn{5}{c}{QPSO}&\multicolumn{5}{c}{SA}\\
\cmidrule(lr){3-7}\cmidrule(lr){8-12}\cmidrule(lr){13-17}
&&S1&S2&S3&$\overline{\epsilon}$&$\overline{t}$&S1&S2&S3&$\overline{\epsilon}$&$\overline{t}$&S1&S2&S3&$\overline{\epsilon}$&$\overline{t}$\\
\midrule
\multirowcell{5}{Griewank}&1&30&30&30&0.000000&0.017&30&30&30&0.000000&0.017&30&30&30&0.000000&0.037\\
&2&30&30&30&0.000000&0.037&30&30&30&0.000000&0.039&30&30&0&0.000436&0.465\\
&3&30&30&30&0.000000&0.115&30&28&16&0.000509&0.064&30&28&0&0.000789&11.711\\
&4&30&30&29&0.000247&1.929&30&18&6&0.004798&0.085&30&2&0&0.006968&17.025\\
&5&30&23&23&0.002218&28.646&30&7&2&0.008428&0.107&30&0&0&0.020390&22.397\\
\hline
\multirowcell{5}{Rosenbrock}&2&30&30&30&0.000000&0.030&30&30&30&0.000000&0.031&0&0&0&$>1$&0.028\\
&3&30&30&30&0.000000&0.121&30&30&30&0.000000&0.047&0&0&0&$>1$&0.038\\
&4&30&23&20&0.018213&2.338&30&23&0&0.001591&0.061&0&0&0&$>1$&0.055\\
&5&30&12&12&0.064526&14.740&30&0&0&0.039457&0.076&0&0&0&$>1$&0.067\\
&6&30&2&2&0.083233&42.835&30&0&0&0.056918&0.090&0&0&0&$>1$&0.080\\
\hline
\multirowcell{5}{Rastrigin}&1&30&30&30&0.000000&0.017&30&30&30&0.000000&0.015&30&30&30&0.000001&0.036\\
&2&30&30&30&0.000000&0.032&30&30&30&0.000000&0.033&30&13&0&0.004223&2.787\\
&3&30&30&30&0.000000&0.052&30&30&30&0.000000&0.049&3&0&0&0.703521&3.052\\
&4&30&30&30&0.000000&0.131&30&30&26&0.000008&0.067&0&0&0&$>1$&3.066\\
&5&30&29&29&0.033165&0.612&30&30&22&0.000024&0.088&0&0&0&$>1$&3.108\\
\hline
\multirowcell{6}{Dixon\&Price}&2&30&30&30&0.000000&2.472&30&30&30&0.000000&0.028&30&24&9&0.000583&0.383\\
&3&30&30&30&0.000000&3.379&30&30&30&0.000000&0.043&0&0&0&$>1$&0.033\\
&4&30&30&30&0.000000&1.515&30&30&30&0.000000&0.055&0&0&0&$>1$&0.049\\
&5&30&30&30&0.000000&2.322&24&24&24&0.133333&0.070&0&0&0&$>1$&0.063\\
&6&30&30&30&0.000000&3.436&12&12&12&0.387794&0.082&0&0&0&$>1$&0.075\\
\hline
\multirowcell{6}{Sphere}&1&30&30&30&0.000000&0.016&30&30&30&0.000000&0.019&30&30&30&0.000000&0.030\\
&5&30&30&30&0.000000&0.076&30&30&30&0.000000&0.081&2&0&0&0.362638&2.089\\
&10&30&30&30&0.000000&0.193&30&30&30&0.000000&0.156&0&0&0&$>1$&1.564\\
&15&30&30&30&0.000000&0.465&30&30&30&0.000000&0.236&0&0&0&$>1$&1.136\\
&20&30&30&30&0.000000&0.972&30&30&30&0.000000&0.315&0&0&0&$>1$&0.721\\
&25&30&30&30&0.000000&2.011&30&30&30&0.000000&0.400&0&0&0&$>1$&0.686\\

\bottomrule
\end{tabular}
\end{table*}






 We know the wavefunction of QPSO is Laplace distribution. Figure \ref{fig:normal} compares the normal distribution and Laplace distribution. Contrast between the two distribution the Laplace distribution is very ``compact''. So QPSO will converge very fast. This will lead to algorithm precocious.

\begin{figure}
\centerline{\includegraphics[width=5cm,angle=-90]{f2.eps} }
\caption{Normal Distribution and Laplace Distribution}
\label{fig:normal}
\end{figure}


\subsection{Iteration times and zero energy at different scale}

Figure \ref{fig:iteration}(a) describes the iteration times at different scale. The main iteration times are concentrated in scale $\sigma_{s}=0.625$ and $\sigma_{s}=0.3125$. We know that Rastrigin function has two scale structures so the iteration times are concertrated in two scale 0.625 and 0.3125. The scales 0.625 and 0.3125 are called ``iteration scale'' by us. The corresponding ground state wavefunctions are described by Figure \ref{fig:iteration}(b). The ground state wavefunction of iteration scale 0.625 is large scale iteration sampling process. This process corresponding to the searching process at large structure $x^2$ of Rastrigin. The ground state wavefunction at iteration scale 0.3125 corresponding to the searching process at small struction $cos(2{\pi}x)$. The iteration scale can help us speculate number and scale of complex test function's structure. Other test function's iteration times at different scale are shown in Table \ref{tab:experiment2}. Table \ref{tab:experiment2} shows that every test function has its own iteration scale.
\begin{figure}
\centering
\subfigure[The iteration times at different scale]{\includegraphics[width=3.5cm,angle=-90]{iteration.eps} }
\subfigure[The ground state wavefunctions of iteration scale]{\includegraphics[width=3.5cm,angle=-90]{scale.eps} }
\caption{The Iteration Times at Different Scale. (The test function is 6 dimensional Rastrigin. $k=20$,$m=200$,$\sigma_{min}=0.000001$. The experiments are repeated 30 times.)}
\label{fig:iteration}
\end{figure}



\begin{table*}
\centering
\caption{Iteration times ($n$) and zero energy ($E^{\sigma_s}_0$) at different scale. Where $k=20$, $n=200$. The iteration times is the average value of 30 experiments.}
\label{tab:experiment2}
\begin{tabular}{c|cccccccccc}
\toprule
\multirow{3}*{Scale}&\multicolumn{10}{c}{Test Function}\\
\cmidrule{2-11}
&\multicolumn{2}{c}{\thead{Griewank\\4D}}&\multicolumn{2}{c}{\thead{Rosenbrock\\4D}}&\multicolumn{2}{c}{\thead{Dixon\&Price\\6D}}&\multicolumn{2}{c}{\thead{Levy\\15D}}&\multicolumn{2}{c}{\thead{Sphere\\20D}}\\
\cmidrule(lr){2-3}\cmidrule(lr){4-5}\cmidrule(lr){6-7}\cmidrule(lr){8-9}\cmidrule(lr){10-11}\\
&$n$&$E^{\sigma_s}_0$&$n$&$E^{\sigma_s}_0$&$n$&$E^{\sigma_s}_0$&$n$&$E^{\sigma_s}_0$&$n$&$E^{\sigma_s}_0$\\
\midrule

20.000000&1&5.985583e-02&1&2.876908e+02&1&3.522336e+02&1&3.325063e+01&1&2.497373e+02\\
10.000000&1&5.985583e-02&1&1.523009e+02&1&6.653597e+01&1&3.234203e+01&1&1.995450e+02\\
5.000000&15&4.025202e-02&1&7.328428e+01&1&6.653597e+01&1&2.433905e+01&1&1.448525e+02\\
2.500000&659&2.062051e-03&1&2.839393e+01&1&1.388395e+01&4&7.652388e+00&3&7.051300e+01\\
1.250000&40&2.062051e-03&1&3.994916e+00&1&1.756383e+00&5&1.457562e+00&3&1.464771e+01\\
0.625000&1&1.911786e-03&2&2.976362e+00&1&9.991064e-01&2&7.568520e-01&3&3.017421e+00\\
0.312500&1&6.098535e-04&3&7.733824e-01&1&7.577751e-01&6&1.754441e-01&4&8.602753e-01\\
0.156250&1&2.084986e-04&11&1.100362e-02&553&1.930900e-01&2&6.366836e-02&4&1.908097e-01\\
0.078125&1&3.814924e-05&21&2.752587e-03&1&3.717616e-02&4&2.173983e-02&3&5.457365e-02\\
0.039062&1&1.910051e-05&22&1.594797e-03&2&4.761271e-03&6&2.405453e-03&4&1.331123e-02\\
0.019531&1&1.316715e-06&37&1.082084e-03&2&1.297243e-03&15&6.494829e-04&4&3.596243e-03\\
0.009766&1&6.869405e-07&35&7.870219e-05&2&3.591595e-04&21&1.177789e-04&3&9.745068e-04\\
0.004883&1&1.591882e-07&29&3.012715e-06&1&8.858013e-05&20&3.369669e-05&3&1.933199e-04\\
0.002441&1&8.318942e-08&17&2.897068e-06&1&8.582513e-06&24&1.067699e-05&3&5.139162e-05\\
0.001221&1&8.855080e-09&16&1.180954e-06&2&6.247687e-06&11&1.644224e-06&3&1.140947e-05\\
0.000610&1&8.855080e-09&23&1.023475e-07&2&1.671792e-06&15&3.351364e-07&3&2.416185e-06\\
0.000305&1&1.945788e-09&25&5.535440e-08&2&5.609016e-07&19&1.939568e-07&3&8.535116e-07\\
0.000153&1&3.094103e-11&32&2.923600e-08&3&4.234581e-08&13&3.785504e-08&3&1.437777e-07\\
0.000076&1&3.094103e-11&27&1.943528e-09&1&1.427265e-08&11&5.612368e-09&4&5.794087e-08\\
0.000038&1&8.539836e-13&35&1.943528e-09&2&6.188260e-09&39&1.572321e-09&4&1.117656e-08\\
0.000019&1&8.539836e-13&53&1.931598e-10&2&3.071223e-10&22&4.733188e-10&3&3.610808e-09\\
0.000010&1&8.539836e-13&69&8.438409e-11&2&3.071223e-10&9&1.471877e-10&3&5.424412e-10\\
0.000005&1&4.599654e-13&158&2.274707e-11&2&1.102464e-10&22&2.497760e-11&3&2.354701e-10\\
0.000002&1&3.785861e-14&116&4.541670e-12&1&3.961683e-12&14&3.963149e-12&3&4.930368e-11\\
0.000001&1&3.375078e-14&234&6.750105e-13&2&3.044706e-12&12&2.186697e-12&3&1.047408e-11\\

\bottomrule
\end{tabular}
\end{table*}



\subsection{Wavefunction Convergence}
The wavefunction acts as an important role in quantum algorithm. For example QPSO uses ground state wavefunction of $\delta$  potential well as sampling probability density function. MQHOA uses wavefunction of Harmonic Oscillator potential as sampling probability density function. The wavefunction reflects the convergence process of algorithm. In this paper
double well potential function is used as objective function to research the convergence process of wavefunction. Double well potential is a kind of classical and simple objective function to be optimized. It can be written as follow

\begin{equation}
f(x)=\nu\frac{(x^2-a^2)^2}{a^4}+\delta\cdot{x}
\end{equation}

Where $\nu$, $a$  and $\delta$  are real constants. $\delta>0$. If $\delta=0$, the function has two minima located at $\pm{a}$, and separeted by a barrier of height $\nu$. If $\delta\cdot{a}\ll\nu$, the two degenerate minima are split by a quantity $\Delta\nu\approx{2}\delta{a}$, the minimum at $x\approx{-a}$  becaming slightly favored.





\begin{figure}
\centerline{\includegraphics[width=5cm,angle=-90]{f1.eps} }
\caption{The Convergence Progress of Wavefunction}
\label{fig:wave}
\end{figure}


In Figure \ref{fig:wave} the parameter values of double well function are $\nu=3.0$ , $a=5.0$  and $\delta=0.1$. Figure \ref{fig:wave} describes one iteration QHO process from high energy to ground state at scale $\sigma_s$. In Figure \ref{fig:wave} the high energy wavefunction implies that the sampling areas include the whole domain. At this stage the tunnel effect is evident. As the energy decreased the wavefunction shrink to ground state wavefunction gradually. According to Figure \ref{fig:wave} the ground state wavefunction looks like a norm distribution. Because the wavefunction is the sampling probability distribution function the ``importance sampling'' areas are shrink to normal distribution at ground state. The algorithm will find the optimal position at the ground state with high probability.  The double well function is a simple objective function it only need one iteration from high energy state to ground state. If the objective function is complex more iteration steps are needed. So the wavefunction can be seemed as the probability distribution of global minimum position. The QHO process makes the sampling centra convergent to the global minimum position. As the QHO process convergents to ground state MQHOA reduces the scale to construct the high energy state at scale $\sigma_s/2$. This process is called M process. The goal of M process is to improve the search accuracy. As $\sigma_s$ reduced the wavefunction shrinks to small areas. When $\sigma_s$ is very small the algorithm will find the optimal position with very high probability.

\section{Conclusions}
We have presented a novel optimization algorithm according to quantum harmonic oscillator model. Some important theoretical problems of MQHOA, such as quantum tunnel effect, uncertainty relation, two scale relation et al, have been researched by this paper. We also have interpreted the reason why the harmonic oscillator potential has been selected by Taylor approximate. Because of the probability interpretation of quantum model the wave functions act as an important role to research the convergence process of MQHOA.

The previous experiments and the experiments in this paper have proved that MQHOA is an excellent optimization algorithm. It is more accurate than QPSO and SA. The convergence process can suit different objective functions automatically.


\section*{Acknowledgment}

This work is supported in part by the National Natural Science Foundation of China (60702075), China Postdoctoral Science Foundation(20070410385, 20090451420) and Key Laboratory of Pattern Recognition and Intelligent Information Processing Foundation in Sichuan Province.


% An example of a floating figure using the graphicx package.
% Note that \label must occur AFTER (or within) \caption.
% For figures, \caption should occur after the \includegraphics.
% Note that IEEEtran v1.7 and later has special internal code that
% is designed to preserve the operation of \label within \caption
% even when the captionsoff option is in effect. However, because
% of issues like this, it may be the safest practice to put all your
% \label just after \caption rather than within \caption{}.
%
% Reminder: the "draftcls" or "draftclsnofoot", not "draft", class
% option should be used if it is desired that the figures are to be
% displayed while in draft mode.
%
%\begin{figure}[!t]
%\centering
%\includegraphics[width=2.5in]{myfigure}
% where an .eps filename suffix will be assumed under latex,
% and a .pdf suffix will be assumed for pdflatex; or what has been declared
% via \DeclareGraphicsExtensions.
%\caption{Simulation results for the network.}
%\label{fig_sim}
%\end{figure}

% Note that IEEE typically puts floats only at the top, even when this
% results in a large percentage of a column being occupied by floats.


% An example of a double column floating figure using two subfigures.
% (The subfig.sty package must be loaded for this to work.)
% The subfigure \label commands are set within each subfloat command,
% and the \label for the overall figure must come after \caption.
% \hfil is used as a separator to get equal spacing.
% Watch out that the combined width of all the subfigures on a
% line do not exceed the text width or a line break will occur.
%
%\begin{figure*}[!t]
%\centering
%\subfloat[Case I]{\includegraphics[width=2.5in]{box}%
%\label{fig_first_case}}
%\hfil
%\subfloat[Case II]{\includegraphics[width=2.5in]{box}%
%\label{fig_second_case}}
%\caption{Simulation results for the network.}
%\label{fig_sim}
%\end{figure*}
%
% Note that often IEEE papers with subfigures do not employ subfigure
% captions (using the optional argument to \subfloat[]), but instead will
% reference/describe all of them (a), (b), etc., within the main caption.
% Be aware that for subfig.sty to generate the (a), (b), etc., subfigure
% labels, the optional argument to \subfloat must be present. If a
% subcaption is not desired, just leave its contents blank,
% e.g., \subfloat[].


% An example of a floating table. Note that, for IEEE style tables, the
% \caption command should come BEFORE the table and, given that table
% captions serve much like titles, are usually capitalized except for words
% such as a, an, and, as, at, but, by, for, in, nor, of, on, or, the, to
% and up, which are usually not capitalized unless they are the first or
% last word of the caption. Table text will default to \footnotesize as
% IEEE normally uses this smaller font for tables.
% The \label must come after \caption as always.
%
%\begin{table}[!t]
%% increase table row spacing, adjust to taste
%\renewcommand{\arraystretch}{1.3}
% if using array.sty, it might be a good idea to tweak the value of
% \extrarowheight as needed to properly center the text within the cells
%\caption{An Example of a Table}
%\label{table_example}
%\centering
%% Some packages, such as MDW tools, offer better commands for making tables
%% than the plain LaTeX2e tabular which is used here.
%\begin{tabular}{|c||c|}
%\hline
%One & Two\\
%\hline
%Three & Four\\
%\hline
%\end{tabular}
%\end{table}


% Note that the IEEE does not put floats in the very first column
% - or typically anywhere on the first page for that matter. Also,
% in-text middle ("here") positioning is typically not used, but it
% is allowed and encouraged for Computer Society conferences (but
% not Computer Society journals). Most IEEE journals/conferences use
% top floats exclusively.
% Note that, LaTeX2e, unlike IEEE journals/conferences, places
% footnotes above bottom floats. This can be corrected via the
% \fnbelowfloat command of the stfloats package.





% if have a single appendix:
%\appendix[Proof of the Zonklar Equations]
% or
%\appendix  % for no appendix heading
% do not use \section anymore after \appendix, only \section*
% is possibly needed

% use appendices with more than one appendix
% then use \section to start each appendix
% you must declare a \section before using any
% \subsection or using \label (\appendices by itself
% starts a section numbered zero.)
%


\appendices

\section{Test functions}

\begin{tabular}{c|l}
\toprule
\multirowcell{2}{Function\\Name}&\multirow{2}*{Function ($n$ is dimension)}\\
&\\
\midrule
Sphere&$f=\sum_{i=1}^{n}x^2_i$\\
\hline
Rosenbrock&$f=\sum_{i=1}^{n-1}[100(x_{i+1}-x^2_i)^2+(1-x_i)^2]$\\
\hline
Rastrigin&$f=10{\cdot}n+\sum_{i=1}^n[x_i^2-10{\cdot}cos(2{\pi}x_i)]$\\
\hline
Griewank&$f=1+\sum_{i=1}^n\frac{x_i^2}{4000}-\prod^n_{i=1}cos\frac{x_i}{\sqrt{i}}$\\
\hline
Dixon\&Price&$f=(x_1-1)^2+\sum^n_{i=2}i(2x_i^2-x_{i-1})^2$\\
\hline
\multirowcell{4}{Levy}&$f=sin^2(\pi\omega_1)$\\
&$+\sum^{n-1}_{i=1}(\omega_i-1)^2[1+10sin^2(\pi\omega_i+1)]$\\
&$+(\omega_n-1)^2[1+sin^2(2\pi\omega_n)]$\\
&where $\omega_i=1+\frac{x_i-1}{4}$\\
\midrule
\end{tabular}
For all test functions the define domain are $[-10,10]$.

% you can choose not to have a title for an appendix
% if you want by leaving the argument blank
\section{}
Appendix two text goes here.




% Can use something like this to put references on a page
% by themselves when using endfloat and the captionsoff option.
\ifCLASSOPTIONcaptionsoff
  \newpage
\fi



% trigger a \newpage just before the given reference
% number - used to balance the columns on the last page
% adjust value as needed - may need to be readjusted if
% the document is modified later
%\IEEEtriggeratref{8}
% The "triggered" command can be changed if desired:
%\IEEEtriggercmd{\enlargethispage{-5in}}

% references section

% can use a bibliography generated by BibTeX as a .bbl file
% BibTeX documentation can be easily obtained at:
% http://www.ctan.org/tex-archive/biblio/bibtex/contrib/doc/
% The IEEEtran BibTeX style support page is at:
% http://www.michaelshell.org/tex/ieeetran/bibtex/
%\bibliographystyle{IEEEtran}
% argument is your BibTeX string definitions and bibliography database(s)
%\bibliography{IEEEabrv,../bib/paper}
%
% <OR> manually copy in the resultant .bbl file
% set second argument of \begin to the number of references
% (used to reserve space for the reference number labels box)

\bibliographystyle{IEEEtran}
\bibliography{IEEEabrv,ref}

% biography section
%
% If you have an EPS/PDF photo (graphicx package needed) extra braces are
% needed around the contents of the optional argument to biography to prevent
% the LaTeX parser from getting confused when it sees the complicated
% \includegraphics command within an optional argument. (You could create
% your own custom macro containing the \includegraphics command to make things
% simpler here.)
%\begin{IEEEbiography}[{\includegraphics[width=1in,height=1.25in,clip,keepaspectratio]{mshell}}]{Michael Shell}
% or if you just want to reserve a space for a photo:
\begin{IEEEbiography}[{\includegraphics[width=1in,height=1.25in,clip,keepaspectratio]{wang}}]{Peng Wang}
received his first bachelor and master degrees in Physics from Sichuan University, Chengdu, China. He obtained the Ph.D degree in Computer Software and Theory from Chinese Academy of Sciences, Chengdu, China, in 2004.

He is a professor of computer science. His research interests include optimization algorithm, quantum algorithm, parellel computing, cloud computing. He is a member of cloud computing professional committee of Chinese Institute of Electronics and high performance computing professional committee of China Computer Federation.
\end{IEEEbiography}


% You can push biographies down or up by placing
% a \vfill before or after them. The appropriate
% use of \vfill depends on what kind of text is
% on the last page and whether or not the columns
% are being equalized.

%\vfill

% Can be used to pull up biographies so that the bottom of the last one
% is flush with the other column.
%\enlargethispage{-5in}



% that's all folks
\end{document}


